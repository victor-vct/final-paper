\documentclass[a4paper, Arial, 12pt]{article}

\usepackage[brazil]{babel}
\usepackage[utf8]{inputenc}
\usepackage{hyperref} % criar hyperlinks
\usepackage{listings} % utilizado para dar highlight em código
\usepackage[dvipsnames]{xcolor}
\usepackage{graphicx} % importar imagens
\usepackage[export]{adjustbox}

\lstdefinestyle{customc}{ %custom color para código do arduino
  belowcaptionskip=1\baselineskip,
  backgroundcolor=\color{lightgray},
  breaklines=true,
  frame=single,
  xleftmargin=\parindent,
  language=C,
  showstringspaces=false,
  basicstyle=\footnotesize\ttfamily,
  keywordstyle=\bfseries\color{ForestGreen},
  commentstyle=\itshape\color{darkgray},
  identifierstyle=\color{blue},
  stringstyle=\color{gray},
  breaklines=true,
  morekeywords={bool},
}

\title{Etapas e dificuldades}
\author{Victor Vieira Paulino
\and
Arthur Cicuto Pires}
\date{\today}

\begin{document}

\maketitle

\section{Módulo do Ponto de Ônibus}

\subsection{Hardware}

\begin{enumerate}
\item HM-10 - Bluetooth 4.0 BLE module
\item Arduino Uno
\end{enumerate}

Arduino Uno é utilizado apenas como ponte para configurar o módulo HM-10. A ligação deve seguir o diagrama abaixo.

\includegraphics[width=10cm, center]{images/arduino-hm10}

\subsection{Software}

\begin{itemize}
\item Arduino IDE 1.8.3 ou superior. 
\end{itemize}

\subsection{Configuração}

Conecte o arduino Uno ao computador e compile o código abaixo utilizando a IDE do arduino.

\lstinputlisting[style=customc]{codes/arduino-code.ino}

Após compilador, utilizando o Serial Monitor da IDE, execute os comandos AT na seguinte ordem:

Obs: Quanto menor o tempo de envio, maior a economia de energia.

\begin{enumerate}
\item AT+RENEW //Coloca nos padrões de fábrica
\item AT+RESET //Reinicia para aplicar os padrões de fábrica
\item AT+MARJ0xNNNN //Define o valor Marjor
\item AT+MINO0xNNNN //Define o valor Minor
\item AT+NAMEMeuBeacon //Define o nome do Beacon
\item AT+ADVI5 //Define tempo de envio. 5 = 546.25 millisegundos
\item AT+ADTY3 //Define como não pareável
\item AT+IBEA1 //Habilita como Beacon
\item AT+DELO2 //Configura para apenas emitir sinal
\item AT+PWRM0 //Habilita auto-sleep para economizar energia
\item AT+RESET
\end{enumerate}


Após configurado, pode ser ligado em uma bateria 3v para utilização.

\subsection{Referências}

\href{ftp://imall.iteadstudio.com/Modules/IM130614001_Serial_Port_BLE_Module_Master_Slave_HM-10/DS_IM130614001_Serial_Port_BLE_Module_Master_Slave_HM-10.pdf}{HM-10 Bluetooth 4.0 BLE module Datasheet}
\\
\href{https://www.arduino.cc/en/main/software}{Arduino IDE}
\\
\href{https://github.com/metractive/beacon-study}{Repositório da Metractive - Como construir Beacons}

\section{Módulo do Ônibus}

\subsection{Hardware}

\begin{itemize}
\item Intel Edison
\item Raspberry Pi 3
\item Tela LCD 7" (em breve)
\item NEO u-blox 6 GPS Modules
\end{itemize}

\subsubsection{Intel Edison}

\includegraphics[width=10cm, center]{images/intel-edison-arduino-kit}

Inicialmente foi adotado o Intel Edison com placa de expansão arduino. Foi escolhido devido a fácil acesso a um exemplar e ótimo hardware. Ele conta com WiFi, Bluetooth, portas I/O, processador Intel Atom de 500 MHz, 1GB de memória RAM DDR3 e 4GB eMMC. \\
Sua utilização foi fácil e não obtivemos nenhuma dificuldade em instalar o sistema que escolhemos.


Problemas encontrados em adotar como solução:
\\

\textbf{Preço}

Embora tenha um ótimo hardware e uma empresa séria por trás da sua construção, o preço, em 07/2017, que gira em torno de R\$ 600,00, não justifica sua adoção como a melhor solução para o projeto já que existe alternativas com preços melhores e bom desempenho.
\\

\textbf{Ausência de controlador gráfico}

Uma das features do projeto é emitir alertas visuais para o motorista por meio de telas LCDs. A placa Intel Edison só nos permite fazer alertas visuais utilizando LEDs e afins. 
\\

\textbf{Descontinuidade da placa pela Intel}

em 07/2017, a Intel anunciou a descontinuidade do desenvolvimento de algumas placas que fabrica. O Intel Edison foi uma delas.

\subsubsection{Raspberry Pi 3}

\includegraphics[width=10cm, center]{images/raspberry-pi}

Testes realizados no Raspberry Pi 3 demonstraram ser uma boa alternativa ao Intel Edison. Foi fácil a instalação do sistema e a placa vem com saída HDMI permitindo utilizar telas LCD para fazer os alertas visuais.
Seu preço, em 08/2017, gira em torno de R\$ 150,00, 1/4 do preço do Intel Edison. Seu hardware contém boas especificações:

\begin{itemize}
\item Quad Core 1.2GHz Broadcom BCM2837 64bit CPU
\item 1GB RAM
\item BCM43438 wireless LAN and Bluetooth Low Energy (BLE) on board
\item 40-pin extended GPIO
\item 4 USB 2 ports
\item 4 Pole stereo output and composite video port
\item Full size HDMI
\item CSI camera port for connecting a Raspberry Pi camera
\item DSI display port for connecting a Raspberry Pi touchscreen display
\item Micro SD port for loading your operating system and storing data
\item Upgraded switched Micro USB power source up to 2.5A
\end{itemize}

Embora ele tenha um hardware com especificações superiores ao Intel Edison, não houve ganho de desempenho ao rodar o sistema, devido a ausência de algoritmos complexos no sistema. Assim, a grande vantagem de se utilizar o Raspberry Pi 3 ao invés do Intel Edison, é seu baixo custo e recurso de chip gráfico.

\subsubsection{NEO u-blox 6 GPS Modules}

\includegraphics[width=10cm, center]{images/neo-6m}

Para realizar o rastreamento do ônibus foi adotado o módulo NEO u-blox 6 GPS Modules, devido a compatibilidade com as placas que contém o sistema embarcado e preço acessível. 

\textbf{Localização}

Uma característica desse módulo é trabalhar com GPS, fazendo comunicação direta com no mínimo 3 satélites para triangular sua posição com mais precisão. Alguns módulos disponíveis no mercado trabalham com A-GPS, que usam torres de telefonia móvel para conhecer sua posição.
O uso do GPS trás maior precisão, porém demora mais para estabelecer conexão com satélites. O A-GPS fornece a localização com menor tempo, porém com menor precisão e a um custo mais alto.

\textbf{Comunicação}

O módulo realiza comunicação UART (Universal Asynchronous Receiver/Transmitter), o que permite fácil comunicação com as placas utilizadas para testes.

\textbf{Preço}

Seu preço, em 08/2017, gira em torno de R\$ 60,00 e pode ser encontrado com facilidade na internet para venda. 


\subsection{Software}

\subsubsection{Sistema Operacional}

\textbf{Android Things}
Em 2016 o Google anunciou o Android Things, uma versão do Android voltada para IoT (Internet of Things). Ele é, atualmente, uma versão do Android Marshmallow reduzida. Sua escolha foi devido a facilidade de embarcar em placas como o Raspberry Pi e Intel Edison, e a variedade de recursos que já estão disponíveis no SO que facilitam o desenvolvimento do módulo, como o recurso LocationManager. \textbf{[Detalhar mais essa parte]}

\subsubsection{IDE}

Foi escolhido o Android Studio como IDE do projeto. Ela é desenvolvida pela IntelliJ e tida pelo Google como ferramenta oficial de desenvolvimento para aplicativos Android.

\subsubsection{Linguagem}

O Google tem duas linguagens de primeiro nível para desenvolvimento Android: Java e Kotlin. Para esse projeto adotamos a linguagem Kotlin, que possui sintaxe muito simplificado em comparação ao Java. Embora Java tenha sido a primeira linguagem oficial para desenvolvimento, Kotlin oferece acesso aos mesmo recursos do sistema.
Algumas bibliotecas disponíveis, desenvolvidas por terceiros, ainda não migraram para o Kotlin, obrigando a implementar algumas classes em Java. Como Kotlin tem interoperabilidade com Java, não existe nenhum impeditivo de utilizar Kotlin e eventualmente alguma classa Java.

\subsection{Referências}

%links do intel edison
\href{https://software.intel.com/en-us/iot/hardware/edison}{Site Oficial Intel Edison}\\
\href{http://download.intel.com/support/edison/sb/edisonmodule_hg_331189004.pdf}{Datasheet Intel Edison}\\
\href{https://www.embarcados.com.br/placas-intel-edison-galileo-e-joule-serao-descontinuadas/}{Anúncio do fim da produção do Intel Edison}\\
%links do raspberry pi
\href{https://www.raspberrypi.org/products/raspberry-pi-3-model-b/}{Site Oficial Raspberry Pi}\\
\href{https://www.raspberrypi.org/documentation/hardware/computemodule/RPI-CM-DATASHEET-V1_0.pdf}{Datasheet Raspberry Pi 3}\\
%links do módulo GPS
\href{https://www.u-blox.com/sites/default/files/products/documents/NEO-6_DataSheet_(GPS.G6-HW-09005).pdf}{Datasheet NEO u-blox 6 GPS Modules}\\
%android things
\href{https://developer.android.com/things/index.html}{Site Oficial Android Things}\\
\href{https://developer.android.com/things/hardware/edison.html}{Configuração do Android Things no Intel Edison}\\
\href{https://developer.android.com/things/hardware/raspberrypi.html}{Configuração do Android Things no Raspberry Pi 3}\\

\section{Aplicativo}

\subsubsection{IDE}

Foi escolhido o Android Studio como IDE do projeto. Ela é desenvolvida pela IntelliJ e tida pelo Google como ferramenta oficial de desenvolvimento para aplicativos Android.

\subsubsection{Linguagem}

O Google tem duas linguagens de primeiro nível para desenvolvimento Android: Java e Kotlin. Para esse projeto adotamos a linguagem Kotlin, que possui sintaxe muito simplificado em comparação ao Java. Embora Java tenha sido a primeira linguagem oficial para desenvolvimento, Kotlin oferece acesso aos mesmo recursos do sistema.
Algumas bibliotecas disponíveis, desenvolvidas por terceiros, ainda não migraram para o Kotlin, obrigando a implementar algumas classes em Java. Como Kotlin tem interoperabilidade com Java, não existe nenhum impeditivo de utilizar Kotlin e eventualmente alguma classa Java.

\section{Web service}

Falar sobre o MEAN stack e sobre as dificuldades que estou encontrando sobre como trabalhar com cada tecnologia da pilha.

\end{document}